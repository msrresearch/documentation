% Generated by Sphinx.
\def\sphinxdocclass{report}
\documentclass[letterpaper,10pt,english]{sphinxmanual}
\usepackage[utf8]{inputenc}
\DeclareUnicodeCharacter{00A0}{\nobreakspace}
\usepackage{cmap}
\usepackage[T1]{fontenc}
\usepackage{babel}
\usepackage{times}
\usepackage[Bjarne]{fncychap}
\usepackage{longtable}
\usepackage{sphinx}
\usepackage{multirow}

\addto\captionsenglish{\renewcommand{\figurename}{Fig. }}
\addto\captionsenglish{\renewcommand{\tablename}{Table }}
\floatname{literal-block}{Listing }



\title{msrresearch Documentation}
\date{July 31, 2015}
\release{0.1}
\author{Björn Guth}
\newcommand{\sphinxlogo}{}
\renewcommand{\releasename}{Release}
\makeindex

\makeatletter
\def\PYG@reset{\let\PYG@it=\relax \let\PYG@bf=\relax%
    \let\PYG@ul=\relax \let\PYG@tc=\relax%
    \let\PYG@bc=\relax \let\PYG@ff=\relax}
\def\PYG@tok#1{\csname PYG@tok@#1\endcsname}
\def\PYG@toks#1+{\ifx\relax#1\empty\else%
    \PYG@tok{#1}\expandafter\PYG@toks\fi}
\def\PYG@do#1{\PYG@bc{\PYG@tc{\PYG@ul{%
    \PYG@it{\PYG@bf{\PYG@ff{#1}}}}}}}
\def\PYG#1#2{\PYG@reset\PYG@toks#1+\relax+\PYG@do{#2}}

\expandafter\def\csname PYG@tok@se\endcsname{\let\PYG@bf=\textbf\def\PYG@tc##1{\textcolor[rgb]{0.25,0.44,0.63}{##1}}}
\expandafter\def\csname PYG@tok@nf\endcsname{\def\PYG@tc##1{\textcolor[rgb]{0.02,0.16,0.49}{##1}}}
\expandafter\def\csname PYG@tok@no\endcsname{\def\PYG@tc##1{\textcolor[rgb]{0.38,0.68,0.84}{##1}}}
\expandafter\def\csname PYG@tok@vg\endcsname{\def\PYG@tc##1{\textcolor[rgb]{0.73,0.38,0.84}{##1}}}
\expandafter\def\csname PYG@tok@kd\endcsname{\let\PYG@bf=\textbf\def\PYG@tc##1{\textcolor[rgb]{0.00,0.44,0.13}{##1}}}
\expandafter\def\csname PYG@tok@sc\endcsname{\def\PYG@tc##1{\textcolor[rgb]{0.25,0.44,0.63}{##1}}}
\expandafter\def\csname PYG@tok@cs\endcsname{\def\PYG@tc##1{\textcolor[rgb]{0.25,0.50,0.56}{##1}}\def\PYG@bc##1{\setlength{\fboxsep}{0pt}\colorbox[rgb]{1.00,0.94,0.94}{\strut ##1}}}
\expandafter\def\csname PYG@tok@si\endcsname{\let\PYG@it=\textit\def\PYG@tc##1{\textcolor[rgb]{0.44,0.63,0.82}{##1}}}
\expandafter\def\csname PYG@tok@kc\endcsname{\let\PYG@bf=\textbf\def\PYG@tc##1{\textcolor[rgb]{0.00,0.44,0.13}{##1}}}
\expandafter\def\csname PYG@tok@nn\endcsname{\let\PYG@bf=\textbf\def\PYG@tc##1{\textcolor[rgb]{0.05,0.52,0.71}{##1}}}
\expandafter\def\csname PYG@tok@gd\endcsname{\def\PYG@tc##1{\textcolor[rgb]{0.63,0.00,0.00}{##1}}}
\expandafter\def\csname PYG@tok@gp\endcsname{\let\PYG@bf=\textbf\def\PYG@tc##1{\textcolor[rgb]{0.78,0.36,0.04}{##1}}}
\expandafter\def\csname PYG@tok@ss\endcsname{\def\PYG@tc##1{\textcolor[rgb]{0.32,0.47,0.09}{##1}}}
\expandafter\def\csname PYG@tok@s2\endcsname{\def\PYG@tc##1{\textcolor[rgb]{0.25,0.44,0.63}{##1}}}
\expandafter\def\csname PYG@tok@sb\endcsname{\def\PYG@tc##1{\textcolor[rgb]{0.25,0.44,0.63}{##1}}}
\expandafter\def\csname PYG@tok@nt\endcsname{\let\PYG@bf=\textbf\def\PYG@tc##1{\textcolor[rgb]{0.02,0.16,0.45}{##1}}}
\expandafter\def\csname PYG@tok@gt\endcsname{\def\PYG@tc##1{\textcolor[rgb]{0.00,0.27,0.87}{##1}}}
\expandafter\def\csname PYG@tok@gh\endcsname{\let\PYG@bf=\textbf\def\PYG@tc##1{\textcolor[rgb]{0.00,0.00,0.50}{##1}}}
\expandafter\def\csname PYG@tok@c\endcsname{\let\PYG@it=\textit\def\PYG@tc##1{\textcolor[rgb]{0.25,0.50,0.56}{##1}}}
\expandafter\def\csname PYG@tok@go\endcsname{\def\PYG@tc##1{\textcolor[rgb]{0.20,0.20,0.20}{##1}}}
\expandafter\def\csname PYG@tok@nb\endcsname{\def\PYG@tc##1{\textcolor[rgb]{0.00,0.44,0.13}{##1}}}
\expandafter\def\csname PYG@tok@nl\endcsname{\let\PYG@bf=\textbf\def\PYG@tc##1{\textcolor[rgb]{0.00,0.13,0.44}{##1}}}
\expandafter\def\csname PYG@tok@mi\endcsname{\def\PYG@tc##1{\textcolor[rgb]{0.13,0.50,0.31}{##1}}}
\expandafter\def\csname PYG@tok@c1\endcsname{\let\PYG@it=\textit\def\PYG@tc##1{\textcolor[rgb]{0.25,0.50,0.56}{##1}}}
\expandafter\def\csname PYG@tok@na\endcsname{\def\PYG@tc##1{\textcolor[rgb]{0.25,0.44,0.63}{##1}}}
\expandafter\def\csname PYG@tok@cp\endcsname{\def\PYG@tc##1{\textcolor[rgb]{0.00,0.44,0.13}{##1}}}
\expandafter\def\csname PYG@tok@sr\endcsname{\def\PYG@tc##1{\textcolor[rgb]{0.14,0.33,0.53}{##1}}}
\expandafter\def\csname PYG@tok@ni\endcsname{\let\PYG@bf=\textbf\def\PYG@tc##1{\textcolor[rgb]{0.84,0.33,0.22}{##1}}}
\expandafter\def\csname PYG@tok@cm\endcsname{\let\PYG@it=\textit\def\PYG@tc##1{\textcolor[rgb]{0.25,0.50,0.56}{##1}}}
\expandafter\def\csname PYG@tok@k\endcsname{\let\PYG@bf=\textbf\def\PYG@tc##1{\textcolor[rgb]{0.00,0.44,0.13}{##1}}}
\expandafter\def\csname PYG@tok@o\endcsname{\def\PYG@tc##1{\textcolor[rgb]{0.40,0.40,0.40}{##1}}}
\expandafter\def\csname PYG@tok@gs\endcsname{\let\PYG@bf=\textbf}
\expandafter\def\csname PYG@tok@bp\endcsname{\def\PYG@tc##1{\textcolor[rgb]{0.00,0.44,0.13}{##1}}}
\expandafter\def\csname PYG@tok@mo\endcsname{\def\PYG@tc##1{\textcolor[rgb]{0.13,0.50,0.31}{##1}}}
\expandafter\def\csname PYG@tok@sh\endcsname{\def\PYG@tc##1{\textcolor[rgb]{0.25,0.44,0.63}{##1}}}
\expandafter\def\csname PYG@tok@nv\endcsname{\def\PYG@tc##1{\textcolor[rgb]{0.73,0.38,0.84}{##1}}}
\expandafter\def\csname PYG@tok@nc\endcsname{\let\PYG@bf=\textbf\def\PYG@tc##1{\textcolor[rgb]{0.05,0.52,0.71}{##1}}}
\expandafter\def\csname PYG@tok@gu\endcsname{\let\PYG@bf=\textbf\def\PYG@tc##1{\textcolor[rgb]{0.50,0.00,0.50}{##1}}}
\expandafter\def\csname PYG@tok@kr\endcsname{\let\PYG@bf=\textbf\def\PYG@tc##1{\textcolor[rgb]{0.00,0.44,0.13}{##1}}}
\expandafter\def\csname PYG@tok@s1\endcsname{\def\PYG@tc##1{\textcolor[rgb]{0.25,0.44,0.63}{##1}}}
\expandafter\def\csname PYG@tok@ge\endcsname{\let\PYG@it=\textit}
\expandafter\def\csname PYG@tok@mb\endcsname{\def\PYG@tc##1{\textcolor[rgb]{0.13,0.50,0.31}{##1}}}
\expandafter\def\csname PYG@tok@il\endcsname{\def\PYG@tc##1{\textcolor[rgb]{0.13,0.50,0.31}{##1}}}
\expandafter\def\csname PYG@tok@mh\endcsname{\def\PYG@tc##1{\textcolor[rgb]{0.13,0.50,0.31}{##1}}}
\expandafter\def\csname PYG@tok@s\endcsname{\def\PYG@tc##1{\textcolor[rgb]{0.25,0.44,0.63}{##1}}}
\expandafter\def\csname PYG@tok@mf\endcsname{\def\PYG@tc##1{\textcolor[rgb]{0.13,0.50,0.31}{##1}}}
\expandafter\def\csname PYG@tok@ne\endcsname{\def\PYG@tc##1{\textcolor[rgb]{0.00,0.44,0.13}{##1}}}
\expandafter\def\csname PYG@tok@sx\endcsname{\def\PYG@tc##1{\textcolor[rgb]{0.78,0.36,0.04}{##1}}}
\expandafter\def\csname PYG@tok@kn\endcsname{\let\PYG@bf=\textbf\def\PYG@tc##1{\textcolor[rgb]{0.00,0.44,0.13}{##1}}}
\expandafter\def\csname PYG@tok@sd\endcsname{\let\PYG@it=\textit\def\PYG@tc##1{\textcolor[rgb]{0.25,0.44,0.63}{##1}}}
\expandafter\def\csname PYG@tok@ow\endcsname{\let\PYG@bf=\textbf\def\PYG@tc##1{\textcolor[rgb]{0.00,0.44,0.13}{##1}}}
\expandafter\def\csname PYG@tok@gr\endcsname{\def\PYG@tc##1{\textcolor[rgb]{1.00,0.00,0.00}{##1}}}
\expandafter\def\csname PYG@tok@err\endcsname{\def\PYG@bc##1{\setlength{\fboxsep}{0pt}\fcolorbox[rgb]{1.00,0.00,0.00}{1,1,1}{\strut ##1}}}
\expandafter\def\csname PYG@tok@vi\endcsname{\def\PYG@tc##1{\textcolor[rgb]{0.73,0.38,0.84}{##1}}}
\expandafter\def\csname PYG@tok@nd\endcsname{\let\PYG@bf=\textbf\def\PYG@tc##1{\textcolor[rgb]{0.33,0.33,0.33}{##1}}}
\expandafter\def\csname PYG@tok@m\endcsname{\def\PYG@tc##1{\textcolor[rgb]{0.13,0.50,0.31}{##1}}}
\expandafter\def\csname PYG@tok@gi\endcsname{\def\PYG@tc##1{\textcolor[rgb]{0.00,0.63,0.00}{##1}}}
\expandafter\def\csname PYG@tok@vc\endcsname{\def\PYG@tc##1{\textcolor[rgb]{0.73,0.38,0.84}{##1}}}
\expandafter\def\csname PYG@tok@kp\endcsname{\def\PYG@tc##1{\textcolor[rgb]{0.00,0.44,0.13}{##1}}}
\expandafter\def\csname PYG@tok@w\endcsname{\def\PYG@tc##1{\textcolor[rgb]{0.73,0.73,0.73}{##1}}}
\expandafter\def\csname PYG@tok@kt\endcsname{\def\PYG@tc##1{\textcolor[rgb]{0.56,0.13,0.00}{##1}}}

\def\PYGZbs{\char`\\}
\def\PYGZus{\char`\_}
\def\PYGZob{\char`\{}
\def\PYGZcb{\char`\}}
\def\PYGZca{\char`\^}
\def\PYGZam{\char`\&}
\def\PYGZlt{\char`\<}
\def\PYGZgt{\char`\>}
\def\PYGZsh{\char`\#}
\def\PYGZpc{\char`\%}
\def\PYGZdl{\char`\$}
\def\PYGZhy{\char`\-}
\def\PYGZsq{\char`\'}
\def\PYGZdq{\char`\"}
\def\PYGZti{\char`\~}
% for compatibility with earlier versions
\def\PYGZat{@}
\def\PYGZlb{[}
\def\PYGZrb{]}
\makeatother

\renewcommand\PYGZsq{\textquotesingle}

\begin{document}

\maketitle
\tableofcontents
\phantomsection\label{index::doc}


Contents:


\chapter{FaceReader}
\label{_static/facereader:welcome-to-facereader-client-s-documentation}\label{_static/facereader::doc}\label{_static/facereader:facereader}

\section{Code documentation}
\label{_static/facereader:code-documentation}\index{FaceReader (class in FaceReader)}

\begin{fulllineitems}
\phantomsection\label{_static/facereader:FaceReader.FaceReader}\pysiglinewithargsret{\strong{class }\code{FaceReader.}\bfcode{FaceReader}}{\emph{host}, \emph{port}, \emph{buffer\_size=4}, \emph{func=None}, \emph{func\_args=None}}{}
FaceReader class to use the API provided by Noldus

The purpose of this class is to operate FaceReader by Noldus. It uses all functionality given by the API documentation from Noldus.
\begin{quote}\begin{description}
\item[{Parameters}] \leavevmode\begin{itemize}
\item {} 
\textbf{\texttt{host}} (\emph{str}) -- IP address of the host running FaceReader

\item {} 
\textbf{\texttt{port}} (\emph{int}) -- Listening port set in FaceReader

\item {} 
\textbf{\texttt{buffer\_size}} (\emph{int}) -- At the moment I have no idea, why I added this parameter or what it is doing.

\end{itemize}

\end{description}\end{quote}
\index{FaceReader.MySocket (class in FaceReader)}

\begin{fulllineitems}
\phantomsection\label{_static/facereader:FaceReader.FaceReader.MySocket}\pysiglinewithargsret{\strong{class }\bfcode{MySocket}}{\emph{host}, \emph{port}, \emph{log=None}, \emph{func=None}, \emph{func\_args=None}, \emph{sock=None}}{}
Class to send and receive messages via TCP.
\index{connect() (FaceReader.FaceReader.MySocket method)}

\begin{fulllineitems}
\phantomsection\label{_static/facereader:FaceReader.FaceReader.MySocket.connect}\pysiglinewithargsret{\bfcode{connect}}{\emph{host}, \emph{port}}{}
Connect to a host via TCP using socket.
\begin{quote}\begin{description}
\item[{Parameters}] \leavevmode\begin{itemize}
\item {} 
\textbf{\texttt{host}} (\emph{str}) -- IP address of host

\item {} 
\textbf{\texttt{port}} (\emph{int}) -- Listening port at host

\end{itemize}

\end{description}\end{quote}

\end{fulllineitems}

\index{receive() (FaceReader.FaceReader.MySocket method)}

\begin{fulllineitems}
\phantomsection\label{_static/facereader:FaceReader.FaceReader.MySocket.receive}\pysiglinewithargsret{\bfcode{receive}}{}{}
Receive a message from the set and connected host.
\begin{quote}\begin{description}
\item[{Returns}] \leavevmode
str - the received message

\end{description}\end{quote}

\end{fulllineitems}

\index{send() (FaceReader.FaceReader.MySocket method)}

\begin{fulllineitems}
\phantomsection\label{_static/facereader:FaceReader.FaceReader.MySocket.send}\pysiglinewithargsret{\bfcode{send}}{\emph{msg}}{}
Send a message.
\begin{quote}\begin{description}
\item[{Parameters}] \leavevmode
\textbf{\texttt{msg}} (\emph{str}) -- Message to be send

\end{description}\end{quote}

\end{fulllineitems}

\index{stop() (FaceReader.FaceReader.MySocket method)}

\begin{fulllineitems}
\phantomsection\label{_static/facereader:FaceReader.FaceReader.MySocket.stop}\pysiglinewithargsret{\bfcode{stop}}{}{}
Stop the thread in which MySocket is running.

\end{fulllineitems}


\end{fulllineitems}

\index{get\_events() (FaceReader.FaceReader method)}

\begin{fulllineitems}
\phantomsection\label{_static/facereader:FaceReader.FaceReader.get_events}\pysiglinewithargsret{\code{FaceReader.}\bfcode{get\_events}}{}{}
Receive a list of all event markers set in the current FaceReader project.
\begin{quote}\begin{description}
\item[{Returns}] \leavevmode
A list of all event markers.

\end{description}\end{quote}

\end{fulllineitems}

\index{get\_stimuli() (FaceReader.FaceReader method)}

\begin{fulllineitems}
\phantomsection\label{_static/facereader:FaceReader.FaceReader.get_stimuli}\pysiglinewithargsret{\code{FaceReader.}\bfcode{get\_stimuli}}{}{}
Receive a list of all stimuli set in the current FaceReader project
\begin{quote}\begin{description}
\item[{Returns}] \leavevmode
A list of all stimuli

\end{description}\end{quote}

\end{fulllineitems}

\index{score\_event() (FaceReader.FaceReader method)}

\begin{fulllineitems}
\phantomsection\label{_static/facereader:FaceReader.FaceReader.score_event}\pysiglinewithargsret{\code{FaceReader.}\bfcode{score\_event}}{\emph{event}}{}
Scores a event marker within the running analysis.
\begin{quote}\begin{description}
\item[{Parameters}] \leavevmode
\textbf{\texttt{event}} (\emph{str}) -- The name of the event marker to be scored.

\end{description}\end{quote}

\end{fulllineitems}

\index{score\_stimulus() (FaceReader.FaceReader method)}

\begin{fulllineitems}
\phantomsection\label{_static/facereader:FaceReader.FaceReader.score_stimulus}\pysiglinewithargsret{\code{FaceReader.}\bfcode{score\_stimulus}}{\emph{stimulus}}{}
Scores a stimuli within the running anlysis
\begin{quote}\begin{description}
\item[{Parameters}] \leavevmode
\textbf{\texttt{stimulus}} (\emph{str}) -- Name of the stimulus to score

\end{description}\end{quote}

\end{fulllineitems}

\index{start\_analyzing() (FaceReader.FaceReader method)}

\begin{fulllineitems}
\phantomsection\label{_static/facereader:FaceReader.FaceReader.start_analyzing}\pysiglinewithargsret{\code{FaceReader.}\bfcode{start\_analyzing}}{}{}
Starts analyzing.

This function sends an action message to FaceReader to start analyzing a preset data source.

\end{fulllineitems}

\index{start\_detailed\_log() (FaceReader.FaceReader method)}

\begin{fulllineitems}
\phantomsection\label{_static/facereader:FaceReader.FaceReader.start_detailed_log}\pysiglinewithargsret{\code{FaceReader.}\bfcode{start\_detailed\_log}}{}{}
Starts a log mode, which returns all of the data known to man.

\end{fulllineitems}

\index{start\_state\_log() (FaceReader.FaceReader method)}

\begin{fulllineitems}
\phantomsection\label{_static/facereader:FaceReader.FaceReader.start_state_log}\pysiglinewithargsret{\code{FaceReader.}\bfcode{start\_state\_log}}{}{}
Starts a log mode, which returns the at the moment dominant facial expression.

\end{fulllineitems}

\index{stop\_analyzing() (FaceReader.FaceReader method)}

\begin{fulllineitems}
\phantomsection\label{_static/facereader:FaceReader.FaceReader.stop_analyzing}\pysiglinewithargsret{\code{FaceReader.}\bfcode{stop\_analyzing}}{}{}
Stops analyzing.

This function sends an action message to Facereader to stop analyzing a preset data source.

\end{fulllineitems}

\index{stop\_detailed\_log() (FaceReader.FaceReader method)}

\begin{fulllineitems}
\phantomsection\label{_static/facereader:FaceReader.FaceReader.stop_detailed_log}\pysiglinewithargsret{\code{FaceReader.}\bfcode{stop\_detailed\_log}}{}{}
Stops the log mode started with {\hyperref[_static/facereader:FaceReader.FaceReader.start_detailed_log]{\emph{\code{start\_detailed\_log()}}}}.

\end{fulllineitems}

\index{stop\_state\_log() (FaceReader.FaceReader method)}

\begin{fulllineitems}
\phantomsection\label{_static/facereader:FaceReader.FaceReader.stop_state_log}\pysiglinewithargsret{\code{FaceReader.}\bfcode{stop\_state\_log}}{}{}
Stops the log mode started with {\hyperref[_static/facereader:FaceReader.FaceReader.start_state_log]{\emph{\code{start\_state\_log()}}}}.

\end{fulllineitems}


\end{fulllineitems}



\section{Logs}
\label{_static/facereader:logs}
At the moment FaceReader supports to kinds of logs: the detailed and
the state log. While the first mentioned includes all data gathered
while analyzing a frame, the later log only contains the at the moment
dominant facial expression.


\subsection{Detailed log}
\label{_static/facereader:detailed-log}
Detailed logging can be enabled by calling the function
\code{start\_detailed\_log()} and can be stopped by calling
\code{stop\_detailed\_log()}.


\subsection{State log}
\label{_static/facereader:state-log}
State logging can be enabled by calling the function
\code{start\_state\_log()} and can be stopped by calling
\code{stop\_state\_log()}.



\renewcommand{\indexname}{Index}
\printindex
\end{document}
